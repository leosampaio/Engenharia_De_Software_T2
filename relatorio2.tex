\documentclass[a4paper,11pt]{article}

\usepackage{fullpage, setspace, url}
\usepackage{colortbl}
\usepackage{color}
\usepackage{amsmath}
\usepackage{indentfirst}
%\usepackage{algorithm,algorithmic}
%\definecolor{G6}{rgb}{0.7,0.7,0.7}
%\newcommand{\fc}{\cellcolor{G6}}
\usepackage{graphicx}
\usepackage{subfigure}
\usepackage{multirow}

% pacotes malucões
\usepackage{graphicx}
\usepackage{amsfonts}
\usepackage{amssymb}
\usepackage{amstext}
\usepackage{hyperref}
\usepackage{ragged2e}
\usepackage{color}
\usepackage{enumerate}
\usepackage{float}
% fonte
\usepackage{helvet}
% coisas de locale
\usepackage[brazil]{babel}
\usepackage[utf8]{inputenc}
\usepackage[T1]{fontenc}


\addtolength{\textwidth}{2cm}
\addtolength{\hoffset}{-1cm}

\addtolength{\textheight}{2cm}
\addtolength{\voffset}{-1cm}

\newcommand{\HRule}{\rule{\linewidth}{0.5mm}}
\newcommand{\DUVIDA}[1]{\huge \textbf {\color{red} #1}\normalsize\\} % escreve uma dúvida grandão vermelha

\begin{document}
% \maketitle

\begin{center}
    \pagestyle{empty} 
    % Upper part of the page
    \textsc{\Large Universidade de São Paulo\\
    Instituto de Ciências Matemáticas e de Computação\\
    SSC0130 - Engenharia de Software - Turma A}\\[5.0cm]

    % Title
    \HRule \\[0.6cm]
    {\Huge Projeto parte 2\\
    Plano de Projeto}\\[0.4cm]
    \HRule \\[3.5cm]

    % Author and supervisor
    \begin{minipage}{0.45\textwidth}
	    \begin{flushleft} \normalsize
    		\emph{Alunos:}
    		
			\[\begin{array}{lr}
	    		\text{Gil Barbosa Reis} & 8532248 \\
    			\text{Giovane Cunha Mocellin} & 8578382 \\
		    	\text{Leonardo Sampaio Ferraz Ribeiro} & 8532300 \\
				\text{Rogério P. Souza} & 5626341
    		\end{array}\]
    	\end{flushleft}
    \end{minipage}
    \begin{minipage}{0.45\textwidth}
	    \begin{flushright} \normalsize
    		\emph{Professora:} \\
    		Ellen Francine Barbosa
	    \end{flushright}
    \end{minipage}

    \vspace{3.0cm}


    \vfill
    % Bottom of the page
    {\Large São Carlos, SP \\ \today}
    \thispagestyle{empty} 
    \newpage
\end{center}

\thispagestyle{plain} 
\setcounter{page}{1}
\setstretch{1.2}
\newpage

\tableofcontents
\newpage

\section{Introdução}
	\subsection{Objetivo do projeto}
		O objetivo deste documento é, entre outros, estabelecer o escopo do projeto, determinar sua viabilidade, analisar os riscos associados, definir recursos necessários, estimar custos e esforço e desenvolver um cronograma do projeto.

	\subsection{Escopo do projeto}
		\subsubsection{Descrição do produto}
			O T-play é  um site voltado para a troca e venda de jogos, tanto físicos, como digitais.
			Os usuários comuns poderão utilizar o sistema para cadastrar ofertas de seus jogos, efetuar a troca de jogos com outros usuários e 
pesquisar ofertas de seu interesse.
			Durante a transação (troca ou venda) entre usuários, pode haver troca de mensagens entre ambos.
			Ao final das transações, usuários devem atribuir uma nota ao outro usuário participante, e uma nota à usabilidade do T-play.
			Há um sistema de ranqueamento que ajuda o usuário durante a escolha do melhor candidato à troca.
			Além disso, caso o usuário queira ofertar um jogo que não exista no sistema, o mesmo pode sugerir o cadastro de tal jogo.
			
			Os administradores poderão utilizar o sistema para gerenciar os patrocinadores e outros administradores, moderar negociações problemáticas, gerenciar cadastro de novos jogos e acessar os diversos relatórios.

			Os patrocinadores possuem os mesmos direitos e restrições que os usuários comuns, mas também recebem, mediante a pagamento em interface exclusiva, o direito de apresentar propagandas no site que redirecionam e enfatizam seus oferecimentos de jogos.

		\subsubsection{Principais entregas do projeto}
			\DUVIDA{Quais ítens devem ter, os do Slide ou os do PDF do trabai?}
			\begin{itemize}
				\item Documento de Requisitos
				\item Plano de projeto (este documento)
				\item Diagrama WBS
				\item Rede PERT com caminho crítico
				\item Tabela de alocação de recursos
				\item Gráfico de Gannt
			\end{itemize}

		\subsubsection{Objetivos do projeto}
			\DUVIDA{Qual a diferença desse pro de cima?}
			O objetivo do projeto é reduzir o esforço e tempo gastos atualmente por pessoas que trocam jogos pela internet e são obrigados a usarem outros meios de comunicação não destinados a esse fim.
			
		\subsubsection{Critérios de aceitação do produto}
			\begin{itemize}
				\item Ter menos de 5\% de avaliações negativas sobre a usabilidade do sistema em transações
				\item 
			\end{itemize}
	
\section{Cronograma de Projeto}
	\subsection{WBS}
	\subsection{Rede PERT com caminho crítico}
	\subsection{Tabela de alocação de recursos}
	\subsection{Gráfico de Gannt}
	
\section{Matriz de Riscos}

\section{Conclusão}


























\end{document} 
